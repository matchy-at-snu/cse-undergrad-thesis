\thispagestyle{empty}
\begin{abstract*}
  \makeatletter
  \begin{center}
    {\Huge \@titlealt }
  \end{center}

  \begin{flushright}
    {\large\@author \\
      College of Natural Sciences \\
      Department of Biological Sciences \\
      Seoul National University \\
    }
  \end{flushright}
  \normalsize
  MPC(Secure Multiparty Computing)는 입력을 각 당사자에게 비공개로 유지하면서 여러 당사자가 계산에 공동으로 기여할 수 있는 방법을 만드는 것을 목표로 하는 암호화 연구 분야입니다. 큰 소수에 대한 곱셈 및 모듈로와 관련된 많은 계산이 필요한 RSA(Rives-Shamir-Adleman) 암호화 알고리즘은 MPC 시나리오에서 작동하도록 수정하기에 적합합니다. 그러나 분산 RSA 키 쌍 생성을 위한 기존 구현이 거의 없으며 복잡한 구성으로 인해 사용도 제한됩니다.\par
  여기에서는 RSA 키 쌍 생성 및 암호 해독을 위한 최신 컨테이너화된 MPC 모듈을 제공합니다. 고성능 RPC(원격 프로시저 호출) 프레임워크인 \texttt{gRPC}를 사용하여 고전적인 Ben-Or, Goldwasser 및 Wigderson(BGW) 프로토콜을 고도의 병렬 방식으로 구현합니다. 구현은 비밀 공유에서 신뢰할 수 있는 딜러의 필요성을 제거한다는 목표를 달성하고 공유 RSA 키 생성의 효율성을 성공적으로 입증했습니다. 체질 방법과 여러 가지 가지치기 기술을 적용하여 기존의 단일 스레드 방식보다 약 50배 더 ​​빠른 성능도 충분히 보여주었습니다.
  \makeatother
\end{abstract*}
\textbf{주요어: } 암호화, 안전다방계산, 분산 RSA 알고리즘, 분산 컴퓨팅, 컨테이너화
\vfill
